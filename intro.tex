\chapter*{Sobre este Documento}

Este resumo tem o objetivo de auxiliar o leitor nos tópicos básicos de Computação. Este documento foi gerado em \today{} às \currenttime. Para contribuições ou acompanhamento do material, acesse o repositório oficial: \url{ainda não existe}. \\

O conteúdo não possuíra uma linguagem específica a ser seguida, as explicações serão em pseudo-código, porém algumas seções podem conter exemplos em C ou Python.

Utilize o apêndice contendo exercícios como complemento. Os exercícios geralmente ajudam a conceituar alguns pontos da matéria por outra perspectiva, mesclando algumas partes do conteúdo. Os exercícios podem ser feitos na linguagem de sua preferência.

Caso você encontre algum erro, seja conceitual, gramático ou ortográfico, envie o problema no repositório do projeto (por meio de \textit{Issue}, ajudando a contribuir para melhorar este documento.

O que ainda não esta pronto será adicionado aos poucos conforme o tempo do escritor.

TODO:
    Conteúdo em si referente a computação básica:
        -Introdução (variáveis, entrada e saída)
        -Estruturas condicionais
        -Estruturas de repetição
        -Ponteiros - Basíco (sem estrutura de dados)
        -Vetores
        -Matrizes
        -Strings
        -Funções
        -Registros (typedef, struct, para python classes simples, dicionarios ou named tuples)
        -Arquivos
        -Recursividade
    Exercícios:
        -Ponteiros - Basíco (sem estrutura de dados)
        -Vetores
        -Matrizes
        -Strings
        -Funções
        -Registros (typedef, struct, para python classes simples, dicionarios ou named tuples)
        -Arquivos
        -Recursividade

Alguns exercícios até repetição já estão escritos, previsão do conteúdo destes até o fim do carnaval.
Bons estudos!