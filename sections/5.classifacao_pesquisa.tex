\chapter{Classificação e Pesquisa}
Quase sempre que se utiliza um computador quer-se pesquisar alguma coisa, porém sem a Classificação do conteúdo a pesquisa se torna muito mais difícil. Da mesma forma que se vai a uma biblioteca para alugar um livro ou filme ou algo do tipo e espera-se que os livros estejam organizados, seja de maneira alfabética ou por genêro entre outros, o conteúdo dos computadores também necessita de ordenação, porém, muitas vezes os valores de um vetor não estarão organizados, portanto torna-se necessário classificar primeiro para depois pesquisar.
\section{Classificação}
Consiste em ordenar os elementos seguindo algum critério. Por exemplo, a ordenação alfabética para dados literais ou crescente/decrescente para dados númericos. Existem diversos algoritmos para executar a ordenação, os mais conhecidos estão entre: ShellSort, InsertionSort, BubbleSort, QuickSort, HeapSort. Neste capítulo focaremos somente no BubbleSort, devido a sua simplicidade, mas vale lembrar que este não será o mais eficiente da lista.
\section{Pesquisa}
Já quanto a pesquisa trataremos da pesquisa sequencial e a pesquisa binária.